\stepcounter{tableCounter} % Increment counter
\setcounter{rowCounter}{0} % Reset counter
\begin{tabularx}{\textwidth}{|>{\columncolor{tableColumnColor}}c|>{\columncolor{tableColumnColor}}c|>{\hsize=1.2\hsize}X|>{\hsize=.8\hsize}X|}
  \hline
  \rowcolor{tableHeaderColor}
  ID & Check & Description & Comments \\ \hline

\multicolumn{4}{|c|}{\cellcolor{tableColumnColor} \textbf{Pre-Firing}} \\ \hline

\cellcolor{cyan}
  \procedureItem{Write down starting time}{Start Time: }
  
  \cellcolor{yellow}
  \procedureItem{Confirm software abort key combination (Ctrl-Space)}{}
  
  \cellcolor{cyan}
  \procedureItem{Define abort responsibilities: 
    \begin{itemize}
      \item 1 pers. (usually SO): surveillance cameras (top left and top right)
      \item 1 pers.: surveillance area (window)
      \item 1 pers. (usually TC): graphs (mainly CC Pressure)
      \item 1 pers. (usually DACS1): UI (sensor values, post-firing/abort)
    \end{itemize}}{The one person responsible for the window is also in charge for starting and stopping the screen recording as well as the recording of the KiDAQ}
  
  \cellcolor{cyan}
  \procedureItem{$\leftrightarrow$TC to all: inform about start of \textbf{Firing} in a few minutes}{}
  
  \cellcolor{yellow}
  \procedureItem{Start video recording with surveillance cameras}{}
  
  \cellcolor{yellow}
  \procedureItem{Check all sensor readings for anomalies:
    \begin{itemize}
      \item Go through all sensors in the UI and read the values out loud, so that everyone in the control station can hear it
    \end{itemize}}{}
  
  \cellcolor{yellow}
  \procedureItem{Perform a surveillance area check}{}
  
  \cellcolor{green}
  \procedureItem{Ensure everyone is in the HUT}{}
  
  \cellcolor{green}
  \procedureItem{→SO announces to everyone that no one is allowed to leave the HUT}{}
  
  \cellcolor{green}
  \procedureItem{Check that window is open}{}
  
  \cellcolor{yellow}
  \procedureItem{Confirm KiDAQ is ready for firing}{}
  
\multicolumn{4}{|c|}{\cellcolor{tableColumnColor} \textbf{Pre-Chill 1}} \\ \hline

\cellcolor{yellow}
  \procedureItem{Switch to PSS plots (if necessary)}{}
  
  \cellcolor{cyan}
  \procedureItem{Wait until OSS Main Line Temperature is -140°C. Monitor OSS Pressure above Tank.}{Indicators showing that the prechill is complete: The time evolution of the temperature in the main line/injector manifold show a drop shortly before the final temperature is reached. You should see ‘clouds’ coming out of the nozzle.}
  
  \cellcolor{yellow}
  \procedureItem{Change phase in the UI to \textbf{INITIAL PRESSURIZATION}}{}
  
  \cellcolor{yellow}
  \procedureItem{Close Main Valve (OSS)}{}
  
  \cellcolor{yellow}
  \procedureItem{→Arm \textbf{}{FIRING} circuit}{}
  
  \cellcolor{yellow}
  \procedureItem{→Disarm \textbf{OSS PRE-FILL} circuit}{}
  
\multicolumn{4}{|c|}{\cellcolor{tableColumnColor} \textbf{Initial Pressurization}} \\ \hline

\cellcolor{cyan}
  \procedureItem{Write down starting time}{Start Time: }
  
  \cellcolor{yellow}
  \procedureItem{Close Vent Valve (FSS)}{}
  
  \cellcolor{yellow}
  \procedureItem{Open Pressurization Valve (FSS PRZ)}{}
  
  \cellcolor{yellow}
  \procedureItem{Check FSS Pressure above Tank}{}
  
  \cellcolor{yellow}
  \procedureItem{Open Pressurization Valve (OSS PRZ)}{}
  
  \cellcolor{yellow}
  \procedureItem{Check OSS Pressure above Tank}{}
  
  \multicolumn{4}{|c|}{\cellcolor{tableColumnColor} \textbf{Pre-CF and Firing}} \\ \hline

  \cellcolor{yellow}
  \procedureItem{Change phase in the UI to \textbf{FIRING}}{}
  
  \cellcolor{yellow}
  \procedureItem{Start recording on KiDAQ}{}
  
  \cellcolor{yellow}
  \procedureItem{Perform a final surveillance area check}{}
  
  \cellcolor{yellow}
  \procedureItem{Switch to Firing plots}{}
  \iftoggle{firing}{
  \cellcolor{green}
  \procedureItem{Turn \textbf{IGNITION KEY} ON}{}
  }{}
  \cellcolor{cyan}
  \procedureItem{$\leftrightarrow$ TC to all: announce start of the \textbf{\iftoggle{firing}{Firing}{Cold Flow}}}{}
  
  \cellcolor{cyan}
  \procedureItem{$\leftrightarrow$TC to all: countdown starts on 5; 
    \begin{itemize}
      \item 5, 4, 3, 2, 1
    \end{itemize}}{"Check after pre-CF:
    \begin{itemize}
      \item OSS Pressure above Tank
      \item OSS pressurization manometer bottle on camera
      \item FSS Pressure above Tank
      \item Mass flow OSS
    \end{itemize}
    Turn on KiDAQ recording after 30s}
  
  \cellcolor{yellow}
  \procedureItem{$\rightarrow$ Initiate \textbf{FIRING} sequence}{}
  
  \cellcolor{cyan}
  \procedureItem{Check that phase in the UI is \textbf{POST-FIRING}:
    \begin{itemize}
      \item If an abort is triggered:
      \begin{itemize}
        \item Check that UI is in \textbf{SAFE STATE}
        \iftoggle{firing}{\item Turn \textbf{IGNITION KEY} OFF}{}
        \item Disarm \textbf{FIRING} circuit
        \item Check system state with surveillance cameras
        \item Check system state with sensor measurements
        \item If system is in a safe state, check which sensor triggered the abort
        \item Analyse and discuss further operations
      \end{itemize}
    \end{itemize}}{
      Possible options on how to continue in case of abort: \\
      \textbf{Overpressure}: consult HEL\_CP\_PSS\_006\_ \\ Overpressure\_01
      \begin{enumerate}
        \item pressurization OSS / FSS $\rightarrow$ check pressure reducers, confirm sensor in depressurized state with gauge
        \item vent OSS: $\rightarrow$ if pressure remains after abort: possibly frozen NO vent valve / SRV: don't approach
        \item CC: check mass flows, injector manifold pressure
      \end{enumerate}
      \textbf{Overtemperature}: Potential damage to engine
      \begin{enumerate}
        \item check water pump functionality (check recording of water pump camera), inspect engine once in safe state
      \end{enumerate}
  }
  
  \cellcolor{green}
  \procedureItem{Turn \textbf{IGNITION KEY} OFF}{}
  
  \cellcolor{green}
  \procedureItem{SO takes \textbf{IGNITION KEY}}{}
  
  \cellcolor{yellow}
  \procedureItem{Check system state with surveillance cameras}{}
  
  \cellcolor{yellow}
  \procedureItem{Check all sensor readings for anomalies:
    \begin{itemize}
      \item Go through all sensors in the UI and read the values out loud, so that everyone in the control station can hear it
    \end{itemize}}{}
  
  \cellcolor{yellow}
  \procedureItem{Stop video recording with surveillance cameras}{}
  
  \cellcolor{yellow}
  \procedureItem{Stop KiDAQ recording}{}
  
  \cellcolor{yellow}
  \procedureItem{Change the phase in the UI to \textbf{RUN TANK DEPRESSURIZATION}}{}
  
\end{tabularx}